\section{Introduction}

\frame
{
  \frametitle{Introduction}
  \framesubtitle{Jean-Marc - 10 to 15 minutes}

  \begin{itemize}
  \item The institutions involved (UFRGS/UFSM/Grenoble/...)
  \item The French-Brazilian collaboration in the domain
  \item The importance of the visualization of parallel applications performance
  \item The tutorial outline
  \end{itemize}
}
\begin{frame}

\frametitle{Collaboration (a not so short story)}
\begin{block}{Academic  work}
\begin{description}
\item Parallel computing in Grenoble (1990)
\item Tracing and monitoring parallel programs
\item Multi-threaded applications
\item Generic visualization tool
\item Distributed Java applications
\item Trace d'architecture
\item System monitoring 
\item ...
\end{description}
\end{block}
\begin{block}{Industrial aspects}
\begin{description}
\item ST microelectronics : embedded systems
\item Bull : Middleware optimization
\item Orange-lab : Distributed computing
\item ...
\end{description}
\end{block}
\end{frame}


\begin{frame}
\frametitle{Analysis of parallel program behavior}
\begin{description}
\item Debugging distributed applications
\item Quantitative analysis (resource utilization analysis) 
\item Performance debugging
\end{description}
\end{frame}

\begin{frame}
\frametitle{Tools for visualization}
State of the art

citer DeRose
\end{frame}

\begin{frame}
\frametitle{Main difficulties}
\begin{description}
\item Large scale
\item Dynamicity of the observed infrastructure
\item Coherence of views
\item Level of abstraction
\end{description}
\end{frame}

\begin{frame}
\frametitle{Typical examples (use case)}
\begin{description}
\item Multi-threaded application : un vieux (pierre-éric) ou un kaapi 
\item Resource usage monitoring : monitoring de grille 
\item Ad-hoc networks (distributed protocol) : travail de Corine
\item 
\item Object based distributed applications : multi-niveuau java
\item Multi-agent systems : exemple paams
\end{description}
\end{frame}
