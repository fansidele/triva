\section{Trace Fundamentals}
\subsection{Fundamentals}
\newcommand\subitem[1]{\\{\scriptsize #1}}
\frame
{
  \frametitle{Performance Analysis}

  \begin{itemize}
     \item{Collect performance data}
     \item{Process collected data}
     \item{Visualize resulting data}
  \end{itemize}
}
\frame
{
  \frametitle{Performance data collection}
  \begin{itemize}
      \item Sampling
\subitem{let the system run, and from time to time, take a look at the state of the system}
%produces execution profiles
      \item Event-driven
\subitem{get informed of interesting changes in system state}
\only<2>{
      \begin{itemize}
      \item Counting
\subitem{count number of times event happened}
      \item Timing
\subitem{accumulate time passed between pairs of events}
      \item Tracing
\subitem{register events for later processing}
\subitem{usually also registers sampling data}
      \end{itemize}
}
  \end{itemize}
}
\frame
{
  \frametitle{Some tracing problems}
  \begin{itemize}
%Techniques for trace collection
  \item Clock synchronization
  \item Timer resolution
  \item Intrusion
\subitem{time / memory / I-O / influence in program behaviour}
  \item Observability
\subitem{level of abstraction}
  \item Matching independently captured events
\subitem{different machines or abstraction levels}
  \item Amount of data
  \item Bufferization
  \item Trace file format
  \end{itemize}
}
\frame
{
  \frametitle{Trace data processing}

  \begin{itemize}
  \item Merge / reorder
  \item Complement information
  \item Filter
  \item Reduce
  \item Prepare data for visualization
  \end{itemize}
}
\subsection{Paj\'e}
\frame
{
\frametitle{Paj\'e} 
\begin{itemize}
\item Generalize visualization tool, remove semantics
\item Trace file contains
\begin{itemize}
\item{hierarchy of containers}
\item{each can contain combination of containers and visualizable entities}
\end{itemize}
\item Entities can contain extra data, used for filtering and reducing; user knows semantics
\item Tool keeps original data and processed data, user chooses views
\end{itemize}
}
\frame
{
\frametitle{Paj\'e} 

\begin{itemize}
\item[] Possible entity types
\begin{itemize}
\item \textbf{event} to represent events that happen at a certain
instant
\item \textbf{state} to represent that
a given container was in a certain state during a certain period of
time
\item\textbf{link} to represent a relation
between two containers that started at a certain instant and
finished at a possibly different instant
\item\textbf{variable} used to represent the evolution in time of a
certain value associated to a container
\end{itemize}
\end{itemize}
}
%  \begin{itemize}
%  \item Language for Trace Description 
%  \item The Paj\'e Simulator and the Visualization Model
%       \begin{itemize}
%       \item[] Container,State,Event,Variable,Link
%       \end{itemize}
%  \item Visual Representation of common patterns with Paj\'e
%    \begin{itemize}
%    \item point-to-point and collective communications
%    \item ...
%    \end{itemize}
%  \end{itemize}
